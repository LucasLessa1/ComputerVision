\documentclass[journal]{IEEEtran}
\usepackage[portuguese]{babel}
\usepackage[utf8]{inputenc}
\usepackage{graphicx}
\usepackage{amsmath}
\usepackage[bottom=2.0cm,top=2.0cm,left=2.0cm,right=2.0cm]{geometry}
\usepackage[portuguese]{babel}
\usepackage{indentfirst}
\usepackage{hyperref}  %%%%
\hypersetup{colorlinks,citecolor=black,filecolor=black,linkcolor=black,urlcolor=black} %%%%
\usepackage{titlesec}
\usepackage{amsmath}
\usepackage[table]{xcolor}
\definecolor{lightgray}{gray}{0.9}
\usepackage{float}
\usepackage{amssymb}
\usepackage{wasysym}
\usepackage{graphicx}
\usepackage[table,xcdraw]{xcolor}
\usepackage{gensymb}
\usepackage{steinmetz}
\DeclareMathOperator{\arctg}{arctg}
\usepackage[pdftex]{hyperref}
\usepackage[center,small]{caption}
\usepackage{titlesec}
\titlespacing*{\section}{0pt}{1.1\baselineskip}{\baselineskip}

%\hyphenation{op-tical net-works semi-conduc-tor}


\begin{document}

\title{Computer Vision and Image Processing Fundamentals}


\author{ Author: Lucas Santos Lessa}

\maketitle

\IEEEpeerreviewmaketitle

\section{Week 2}

A digital image can be interpreted as a rectangulas array of numbers.
A gray-scale image is an image that is made up of different shades of gray.\\
If we zoom into the region,
we see the image is comprised of
a rectangular grid of blocks called pixels.\\
Digital images have intensity values between 0 (Black) and 255 (White).
How digital image is a matrix, for rows, we start at the top of the image and move down.
For columns, we start at
the left of the image and move right.
\\
Color values are represented as different channels.
Like the gray-scale image,
each channel is an image.
\\
A color image is like a cube.
\\
In addition to accessing
each image intensity with a row and column index,
we also have an index for each channel, in this case,
zero for red, one for blue and two for green.
\\
\\
The pillow, or PIL library is
a popular library for working with images in Python.
\\
\\
OpenCV is a library used for computer vision.
\\
\\

\ifCLASSOPTIONcaptionsoff
  \newpage
\fi

% that's all folks
\end{document}
